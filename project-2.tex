\documentclass{article}
\author{Monnier Killian\and André Émilien}
\title{Project 2: Explainable Artificial Intelligence and Cybersecurity}
\begin{document}
\maketitle
\section{Introduction}
Neural networks with fine-tuned weights outperform ML algorithms but their results are difficult to interpret, and, one cannot manually define a rule in the prediction process.\par
This lack of interpretability prevents banks and other financial institutions from using them.\par
Explainable AI is a field of study that aims to address these issues.\par
Another advantage of XAI is when rules are defined it makes it impossible for training biases.\par
If an Intrusion Detection System (IDS) use XAI in Security Information and Event Management (SIEM). A SIEM analyst will have explainations on detected threats to take important decisions such as shutting down the Information System (IS).\par
The use case we are dealing with in this project is the detection of breast cancers. Sensitivity ranges between 53.1 and 73\% among radiologists and specificity is 96\%. [2][3]
\section{State of the Art}
Current explainable IA methods are SHAP and BRCG. Protodash shows samples from training dataset which are similar to given sample, it shows similarities and differences between them, etc. For end user LIME, SHAP and CEM explain which features in the input instance are contributing in model's final decision and how model's decision can be changed by tweaking their values. [1]
\section{References}
[1] : Mane, Shraddha \& Rao, Dattaraj. (2021). Explaining Network Intrusion
Detection System Using Explainable AI Framework.
[2] : Britton P, Warwick J, Wallis MG, O'Keeffe S, Taylor K, Sinnatamby R, Barter S, Gaskarth M, Duffy SW, Wishart GC. Measuring the accuracy of diagnostic imaging in symptomatic breast patients: team and individual performance. Br J Radiol. 2012 Apr;85(1012):415-22. doi: 10.1259/bjr/32906819. Epub 2011 Jan 11. PMID: 21224304; PMCID: PMC3486650.
[3] : Mattie Salim , Karin Dembrower, Martin Eklund, Peter Lindholm, Fredrik Strand. Range of Radiologist Performance in a Population-based Screening Cohort of 1 Million Digital Mammography Examinations.
\end{document}
